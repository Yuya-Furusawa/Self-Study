\documentclass[12pt]{article}
\usepackage[affil-it]{authblk}
%
\usepackage{amsmath,amssymb,amsthm,ascmac}
\usepackage{float}
\usepackage[dvipdfmx]{graphicx}
\usepackage{subcaption}
\usepackage{here}
\usepackage{pifont}
\usepackage{comment}
\usepackage{tikz}
\usepackage{geometry}
\usepackage{pgfplots}
\usepackage{dsfont}
\usepackage{bm}
\usetikzlibrary{patterns}

\usepackage{mathptmx} % rm & math
\usepackage[scaled=0.90]{helvet} % ss
\usepackage{courier} % tt
\normalfont
\usepackage[T1]{fontenc}

\geometry{left=1.0in,right=1.0in,top=1.0in,bottom=1.0in}
%¥usepackage[top=25truemm,bottom=25truemm,left=10truemm,right=10truemm]{geometry}
\usepackage{setspace}
\setstretch{1.5} %行間の広さ指定

\theoremstyle{definition}
\newtheorem{theorem}{Theorem}
\newtheorem*{theorem*}{Theorem}
\newtheorem{lemma}{Lemma}
\newtheorem*{lemma*}{Lemma}
\newtheorem{proposition}{Proposition}
\newtheorem*{proposition*}{Proposition}
\newtheorem{definition}{Definition}
\newtheorem*{definition*}{Definition}
\newtheorem{corollary}{Corollary}
\newtheorem*{corollary*}{Corollary}
\newtheorem{assumption}{Assumption}
\newtheorem*{assumption*}{Assumption}
\newtheorem{example}{Example}
\newtheorem*{example*}{Example}
\newtheorem*{remark}{Remark}
\newtheorem{algorithm}{Algorithm}
\newtheorem*{algorithm*}{Algorithm}

\newcommand{\bs}[1]{\boldsymbol{#1}}


\begin{document}

\title{Network Game with Endogenous Formation and Peer-Effect}

\author{Yuya Furusawa}
\affil{U-Tokyo, GSE, 29-186036}

\date{\today}

\maketitle

\begin{abstract}
aaaa
\end{abstract}


\section{Introduction}

TO DO
\begin{itemize}
	\item Network with peer effects
	\item Importance of network formation
	\begin{itemize}
 		\item Model Story
 		\item Numerical example
 	\end{itemize}
 	\item Summary of results
  	\item Literature Review
\end{itemize}


\section{Model}

There are $n$ agents in the economy, and denote the set of agents $N = \{ 1, \cdots, n\}$.
We assume $n$ is finite and $n \ge 2$.

The game consists of two-stage game: "choosing neighbors" and "choosing an effort level".
Initially, agents are connected in the potential network $g^p$.
We consider the network $g^p$ is connected, directed and unweighted.
The potential network $g^p$ is represented by adjacency matrix $\bm{G}^p = {(g_{ij}^p)}_{ij}$ where, for any $i \neq j$,
\[ g_{ij}^p =
	\begin{cases}
		1 \  (\text{if $i$ has a link to $j$ in $g^p $}) \\
		0 \  (\text{otherwise})
	\end{cases} \]
Note that $\bm{G}^p$ can be asymetric and $g_{ii}^p = 0$ for all $i$.
Denote the set of agent $i$'s neighbors in the potential network $N_i(g^p)$.
In the first stage, each agent $i$ chooses the partners from the potential neighbors, $N_i(g^p)$.
This strategy can be represented by the function $\phi_i : N \rightarrow N$ for all $i$.
When forming a link, each agent incurs a cost $c_{ij} \ge 0$.
After forming links, the network $g$ is realized.
$g$ is represented by the adjacency matrix $\bm{G} = {(g_{ij})}_{ij}$ such that, for any pair $(i,j)$,
\[ g_{ij} = 
	\begin{cases}
		1 \  (\text{if} \  j \in \phi_i(N_i(g^p)) ) \\
		0 \  (\text{otherwise})
	\end{cases} \]
We can see the realization of the network depends on the potential networl $g^p$ and costs $\bm{C} = {(c_{ij})}_{ij}$.
To represent this dependency, we write the realized network $g$ as $g(g^p, C)$.

In the second stage, each agent $i = 1, \cdots, n$ excerts an effort $x_i \ge 0$, and gets a payoff which depends on the agents' efforts and realized network.
Denote $\bm{x} = (x_1, \cdots, x_n)$.
\[ u_i(\bm{x}, g, \bm{C}) = v_i(\bm{x}, g, \phi) - \sum_{j=1} g_{ij} c_{ij} \]
where
\[ v_i(\bm{x}, g, \phi) = \alpha_i x_i - \frac{1}{2} x_i^2 + \phi \sum_{j=1}^n g_{ij} x_i x_j \]
Here, we focus on bilinear payoff functions.
We assume $\alpha_i > 0$ for all $i$, and $\phi > 0$.
Denote $\bm{\alpha} = (\alpha_1, \cdots, \alpha_n)$.

Before go into an equilibrium analysis, we introduce a well-known network centrality measure which is useful for our analysis.
Consider the $n$-square adjacency matrix $\bm{G}$ of a network $g$ where $g_{ij} = 1$ if $i$ has a direct link to $j$ and $g_{ij} = 0$ otherwise.
We introduce a following definition from Katz(1953) and Bonacich(1987).

\begin{definition}
Given a vector $\bm{u} \in \mathbb{R}_+^n$, and a small enough $\phi \ge 0$, the vector of Katz-Bonacich centralities of parameter $\phi$ in network $g$ is defined as:

\[ \bm{b}_{\bm{u}} = {(\bm{I} - \phi \bm{G})}^{-1} \bm{u} = \sum_{p=0}^{\infty} \phi^p \bm{G}^p \bm{u} \]

\end{definition}

Katz-Bonacich centrality can be thought of a centrality measure which considers the discounted sum of the walks.
The $i$-th entry of the vector $\bm{b}_{\bm{u}}$ is denoted by $\bm{b}_{\bm{u}, i}$.
We write the sum of Katz-Bonacich centrarities as $b_{\bm{u}} = \sum_{i=1}^n \bm{b}_{\bm{u}, i}$


\section{Equilibrium}

In this paper, we consider the subgame perfect equilibrium where all agents take pure strtegies in every subgame.
The reason why we consider such a equilibrium is that pure strategies in choosing neighbors makes network deterministic.
If agents take mixed actions in the first stage, the realized network can be stochastic and we cannot directly compare the realized network.

We characterize the equilibrium by standard way: backward induction.
First, given realized network $g$, consider the second stage, choosing an effort level.
To characterize the equilibrium in this subgame, we require a new assumption.
Denote $\rho (\bm{G})$ as the spectral radius of the matrix $\bm{G}$.

\begin{assumption}
$\phi \rho(\bm{G}^p) < 1$
\end{assumption}

We show that, under this assumption, there exists a unique Nash equilibrium in the second stage for any realized network.

\begin{proposition}
Under Assumption 1, for any realized network $G$, the subgame has a unique Nash equilibrium $\bm{x}^*$, which is interior and given by
\[ \bm{x}^* = {(\bm{I} - \phi \bm{G})}^{-1} \bm{\alpha} \]
\end{proposition}

Ballester, Calvo-Armengol, and Zenou(2006, 2010) show that there exists a unique interior Nash equilibrium when the spectral radius of the adjacency matrix of the network is small enough.
In our model, the spectral radius of the adjacency matrix of realized network may not be satisfy that condition.
Proposition 1 says that we can satisfy the condition under Assumption 1.
Note that Proposition 1 also provides the Nash equilibrium, which is a first step of backward induction.

In order to characterize the subgame perfect equilibrium, next we consider the agents' strategy to choose neighbors.
In light of Proposition 1, we only need to consider Nash equilibria given the efforts in the second stage Nash equilibrium.
Since the number of agents and strategies is finite, there exists Nash equilibria.
As mentioned above, we will consider the pure staretegy equilibrium.
To show the existence of such a equilibrium, we consider the following algorithm.

\begin{algorithm}
\ 
\begin{description}
	\item[Step 0.]\mbox{}\\
		Let $G^{(0)}$ be the initial realized network where $G^{(0)} = G^p$.
		Compute each players' payoffs.
		Denote the set of agents who do not take best response as $NB^{(0)}$.
		Go into Step 1.
	\item[Step $k(\ge 1)$.]\mbox{}\\
		Check whether $NB^{(k-1)}$ is empty or not.
		If $NB^{(k-1)} = \emptyset$, define $G^* = G^{(k-1)}$ and terminate the algorithm.
		Otherwise, choose a agent $i \in NB^{(k-1)}$ randomly.
		Agent $i$ takes best response for $G^{(k-1)}$.
		After $i$'s changing links, new network is emerged and let such a network $G^{(k)}$.
		Compute each players' payoffs and define $NB^{(k)}$.
		Proceed Step $k+1$.
\end{description}
\end{algorithm}

Algorithm 1 is one of the "best response dynamics" algorithms, where agents take their best reponses in each steps.
To show the theorem, we need some additional lemmas.

\begin{lemma}
	$v_i^* (\hat{g}, \bm{x}) > v_i^* (g, \bm{x})$
\end{lemma}

\begin{lemma}
	$v_i(\tilde{g_1}, \bm{x}) - v_i(g_1, \bm{x}) \ge v_i(\tilde{g_2}, \bm{x}) - v_i(g_2, \bm{x})$
\end{lemma}

Using Algorithm 1 and these lemmas, we can show there exists a pure strategy subgame perfect equilibrium.

\begin{theorem}
Algorithm 1 terminates in finite steps, and converged network represents the equilibrium network
\end{theorem}

The algorithm never fails to converge because the removed links will not be reformed in the later steps.
The intuition behind this theorem is that

Note that the equilibrium may not be unique.
Consider the following example.

\begin{example}
Consider the network with $n=2$, and the potential network as,
\[ G^p = \left[
	\begin{array}{cc}
		0 & 1 \\
		1 & 0
	\end{array} \right] \]
Assume $\bm{\alpha} = (1, 1)$ and $\phi$ satisfies Assumption 1.
Denote the realized network $g$ as
\[ G = \left[
	\begin{array}{cc}
		0 & g_{12} \\
		g_{21} & 0
	\end{array} \right] \]
where $g_{12}, g_{21} \in \{0,1\}$.
By proposition 1, we can compute each player's payoffs.
\[ (u_1, u_2) =
	\begin{cases}
		\left( \frac{1}{2}{\left( \frac{1}{1 - \phi} \right)}^2 - c_{12}, \frac{1}{2}{\left( \frac{1}{1 - \phi} \right)}^2 - c_{21} \right) & (g_{12} = 1, g_{21} = 1) \\
		\left( \frac{1}{2}{(1+\phi)}^2 - c_{12}, \frac{1}{2} \right) & (g_{12} = 1, g_{21} = 0) \\
		\left( \frac{1}{2}, \frac{1}{2}{(1+\phi)}^2 - c_{21} \right) & (g_{12} = 0, g_{21} = 1) \\
		\left( \frac{1}{2}, \frac{1}{2} \right) & (g_{12} = 0, g_{21} = 0)
	\end{cases} \]
From this calculation, if $c_{12}$ and $c_{21}$ satisfy
\[ \frac{1}{2}{(1+\phi)}^2 - \frac{1}{2} < c_{12} < \frac{1}{2}{\left( \frac{1}{1 - \phi} \right)}^2 \]
\[ \frac{1}{2}{(1+\phi)}^2 - \frac{1}{2} < c_{21} < \frac{1}{2}{\left( \frac{1}{1 - \phi} \right)}^2 \]
then $(g_{12}, g_{21}) = (1,1)$ and $(g_{12}, g_{21}) = (0,0)$ are both Nash equilibria.
Thus, $G = \left[
	\begin{array}{cc}
		0 & 1 \\
		1 & 0
	\end{array} \right]$
and $G = \left[
	\begin{array}{cc}
		0 & 0 \\
		0 & 0
	\end{array} \right]$
are both equilibrium networks.
\end{example}

Example 1 shows the equilibrium network may not be unique.
From this perspective, the algorithm can potentially converge multiple equilibria.
In each step, the number of the agents who do not take best responses is generally greater than one.
Therefore, the order of the agents who delete the links is changed, and the network in each steps is also changed.
This may lead to the different networks.
However, we can show that the equilibrium which is derived by the algorithm is unique.

\begin{theorem}
Given $C, G^p, \bm{\alpha}, and \phi$, the equilibrium derived by Algorithm 1 is unique.
\end{theorem}

From this theorem, the order of the agents who take best responses does not matter.
The set of deleted links is determined only by costs, but not by the other players' links.


\section{Comparative Statics}


\section{Finding the Key Player}


\section{Conclusion}


\section{Reference}


\section{Appendix}


\end{document}
