\documentclass[12pt]{article}
\usepackage[affil-it]{authblk}
%
\usepackage{amsmath,amssymb,amsthm,ascmac}
\usepackage{float}
\usepackage[dvipdfmx]{graphicx}
\usepackage{subcaption}
\usepackage{here}
\usepackage{pifont}
\usepackage{comment}
\usepackage{tikz}
\usepackage{geometry}
\usepackage{pgfplots}
\usepackage{dsfont}
\usetikzlibrary{patterns}

\usepackage{mathptmx} % rm & math
\usepackage[scaled=0.90]{helvet} % ss
\usepackage{courier} % tt
\normalfont
\usepackage[T1]{fontenc}

\geometry{left=1.0in,right=1.0in,top=1.0in,bottom=1.0in}
%¥usepackage[top=25truemm,bottom=25truemm,left=10truemm,right=10truemm]{geometry}
\usepackage{setspace}
\setstretch{1.5} %行間の広さ指定

\theoremstyle{definition}
\newtheorem{theorem}{Theorem}
\newtheorem*{theorem*}{Theorem}
\newtheorem{lemma}{Lemma}
\newtheorem*{lemma*}{Lemma}
\newtheorem{proposition}{Proposition}
\newtheorem*{proposition*}{Proposition}
\newtheorem{definition}{Definition}
\newtheorem*{definition*}{Definition}
\newtheorem{corollary}{Corollary}
\newtheorem*{corollary*}{Corollary}
\newtheorem{assumption}{Assumption}
\newtheorem*{assumption*}{Assumption}
\newtheorem{example}{Example}
\newtheorem*{example*}{Example}
\newtheorem*{remark}{Remark}

\newcommand{\bm}[1]{{\mbox{\boldmath $#1$}}}


\begin{document}

\title{Endogenous Network Formation with Peer Effects}

\author{Yuya Furusawa}
\affil{U-Tokyo, GSE, 29-186036}

\date{\today}

\maketitle

\begin{abstract}
aaaa
\end{abstract}


\section{Introduction}

TO DO
\begin{itemize}
  \item Model Story
  \item Literature Review
\end{itemize}


\section{Model}

There are $n$ agents in the economy, and denote the set of agents $N = \{ 1, \cdots, n\}$.
We assume $n$ is finite and $n \ge 2$.

The game consists of two-stage game: "choosing neighbors" and "choosing an effort level".
Initially, agents are connected in the potential network $g^p$.
We consider the network $g^p$ is connected, directed and unweighted.
The potential network $g^p$ is represented by adjacency matrix $G^p = {(g_{ij}^p)}_{ij}$ where, for any $i \neq j$,
\[ g_{ij}^p =
	\begin{cases}
		1 \  (\text{if $i$ has a link to $j$ in $g^p $}) \\
		0 \  (\text{otherwise})
	\end{cases} \]
Note that $G^p$ can be asymetric and $g_{ii}^p = 0$ for all $i$.
Denote the set of agent $i$'s neighbors in the potential network $N_i(g^p)$.
In the first stage, each agent $i$ chooses the partners from the potential neighbors, $N_i(g^p)$.
This strategy can be represented by the function $\phi_i : N \rightarrow N$ for all $i$.
When forming a link, each agent incurs a cost $c_{ij} \ge 0$.
After forming links, the network $g$ is realized.
$g$ is represented by the adjacency matrix $G = {(g_{ij})}_{ij}$ such that, for any pair $(i,j)$,
\[ g_{ij} = 
	\begin{cases}
		1 \  (\text{if} \  j \in \phi_i(N_i(g^p)) ) \\
		0 \  (\text{otherwise})
	\end{cases} \]
We can see the realization of the network depends on the potential networl $g^p$ and costs $C = {(c_{ij})}_{ij}$.
To represent this dependency, we write the realized network $g$ as $g(g^p, C)$.

In the second stage, each agent $i = 1, \cdots, n$ excerts an effort $x_i \ge 0$, and gets a payoff which depends on the agents' efforts and realized network.
\[ u_i(x_1, \cdots, x_n, g, C) = v_i(x_1, \cdots, x_n, g, \phi) - \sum_{j=1} g_{ij} c_{ij} \]
where
\[ v_i(x_1, \cdots, x_n, g, \phi) = \alpha_i x_i - \frac{1}{2} x_i^2 + \phi \sum_{j=1}^n g_{ij} x_i x_j \]
Here, we focus on bilinear payoff functions.
We assume $\alpha_i > 0$ for all $i$, and $\phi > 0$.
Denote $\boldsymbol{x} = (x_1, \cdots, x_n)$ and $\boldsymbol{\alpha} = (\alpha_1, \cdots, \alpha_n)$

\section{Equilibrium}


\section{Comparative Statics}


\section{Finding the Key Player}


\section{Conclusion}


\section{Reference}


\end{document}
